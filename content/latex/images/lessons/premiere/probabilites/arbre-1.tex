\documentclass[tikz]{standalone}
\usepackage{fourier}
\usepackage{tikz}

\begin{document}
	% Default drawing color
	\providecommand{\tikzcolor}{black}
	
	% Set the overall layout of the tree
	\tikzstyle{level 1}=[level distance=3.5cm, sibling distance=3.5cm]
	\tikzstyle{level 2}=[level distance=3.5cm, sibling distance=2cm]
	
	% Define styles for bags and leafs
	\tikzstyle{bag} = [text width=4em, text centered]
	\tikzstyle{end} = [circle, minimum width=3pt,fill, inner sep=0pt]
	
	% The sloped option gives rotated edge labels. Personally
	% I find sloped labels a bit difficult to read. Remove the sloped options
	% to get horizontal labels.
	\begin{tikzpicture}[grow=right, sloped, draw=\tikzcolor, text=\tikzcolor, fill=\tikzcolor]
		\node[bag] {$\Omega$}
		child {
			node[bag] {$\bar{A}$}
			child {
				node[end, label=right:
				{$\bar{B}$}] {}
				edge from parent
				node[above] {$P_{\bar{A}}(\bar{B})$}
				node[below]  {}
			}
			child {
				node[end, label=right:
				{$B$}] {}
				edge from parent
				node[above] {$P_{\bar{A}}(B)$}
				node[below]  {}
			}
			edge from parent
			node[above] {$P(\bar{A})$}
			node[below]  {}
		}
		child {
			node[bag] {$A$}
			child {
				node[end, label=right:
				{$\bar{B}$}] {}
				edge from parent
				node[above] {$P_{A}(\bar{B})$}
				node[below]  {}
			}
			child {
				node[end, label=right:
				{$B$}] {}
				edge from parent
				node[above] {$P_{A}(B)$}
				node[below]  {}
			}
			edge from parent
			node[above] {$P(A)$}
			node[below]  {}
		};
	\end{tikzpicture}
\end{document}